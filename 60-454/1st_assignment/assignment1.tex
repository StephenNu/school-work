\documentclass{article}
\usepackage[english]{babel}
\usepackage{amsmath}
\usepackage{amssymb}
\usepackage{changepage}
\usepackage{amsthm}
\usepackage[margin=1in]{geometry}
\usepackage{fancyhdr}

\pagestyle{fancy}
\rhead{Stephen Nusko \\ 103693282}
\lhead{0360454 \\ Assignment 1}
\title{[0360454] Assignment 1}
\author{Stephen Nusko \\ 103693282}

\begin{document}
	\thispagestyle{empty}
	\maketitle	
  \section{Question 1}
    let $f: \mathbb{N} \rightarrow \mathbb{R}^{+}\cup \{0\}$ and $g: \mathbb{N} \rightarrow \mathbb{R}^{+} \cup \{0\}$. Prove that $O(f) \subset O(g)$ if and only if $f \in O(g)$ and $f \not\in \Omega(g)$ \\
    We need to prove that $O(f) \subset O(g) \Leftrightarrow  f \in O(g) \land f \not\in \Omega(g)$ \\
    \subsection{Left to right}
      First we will prove $O(f) \subset O(g) \Rightarrow f \in O(g) and f \not\in \Omega(g)$ \\
      Proof: (contradiction) \\
      Assume $O(f) \subset O(g)$ and $(f \not\in O(g) \lor f \in \Omega(g))$ \\
      \subsubsection{Case 1: $f \not\in O(g)$}
      \begin{enumerate}
        \item $f \in O(f)$ \hfill Theorem 0.3 (a)
        \item $f \in O(g)$ \hfill $O(f) \subset O(g)$
        \item $f \in O(g) \land f \not\in O(g)$ \hfill 60-231
        \item $false$
      \end{enumerate}
      \subsubsection{Case 2: $f \in \Omega(g)$}
      \begin{enumerate}
          \item $(\exists z: \mathbb{N} \rightarrow \mathbb{R}^{+} \cup \{0\})z \in O(g) \land z \not\in O(f)$ \hfill $O(f) \subset O(g)$
          \item $z \in O(g) \land z \not\in O(f)$ \hfill EI
          \item $z \in O(g)$ and $z \not\in O(f)$ \hfill I2, E9, I2
          \item $c_{1}g(n) \leq f(n)$  $\forall n \geq n_{0}$ \hfill $f \in \Omega(g)$
          \item $g(n) \leq \frac{1}{c_{1}}f(n)$  $\forall n \geq n_{0}$ \hfill Math
          \item $z(n) \leq c_{2}g(n)$  $\forall n \geq n_{1}$ \hfill $z \in O(g)$
          \item $\frac{1}{c_{2}}z(n) \leq g(n)$  $\forall n \geq n_{1}$ \hfill Math
          \item $\frac{1}{c_{2}}z(n) \leq \frac{1}{c_{1}}f(n)$  $\forall n \geq max(n_{0}, n_{1})$ \hfill transitivity
          \item $z(n) \leq \frac{c_{2}}{c_{1}}f(n)$  $\forall n \geq max(n_{0}, n_{1})$ \hfill Math
          \item $z(n) \in O(f)$ \hfill Definition of $O(f)$
          \item $z(n) \in O(f) \land z(n) \not\in O(f)$ \hfill 60-231
          \item $false$
      \end{enumerate}
    \subsection{Right to left}
      Next we prove $f \in O(g) \land f \not\in \Omega(g) \Rightarrow O(f) \subset O(g)$ \\
      We will prove that $O(f) \subseteq O(g)$ and then $O(g) \neq O(f)$ \\
      Assume $f \in O(g)$ and $f \not\in \Omega(g)$
      \subsubsection{Prove: $O(f) \subseteq O(g)$}
      Proof: (direct) \\
      We are to prove that $(\forall z)(z \in O(f) \Rightarrow z \in O(g)$
          \begin{enumerate}
            \item let $z \in O(f)$ \hfill assumption
            \item $z \in O(f) \land f \in O(g)$ \hfill 60-231
            \item $z \in O(g)$ \hfill Theorem 0.3 (c)
          \end{enumerate}
          $\therefore O(f) \subseteq O(g)$
      \subsubsection{Prove: $O(g) \neq O(f)$}
        Proof: (contradiction)
        \begin{enumerate}
          \item Suppose $O(g) = O(f)$ \hfill assumption
          \item $g \in O(g)$ \hfill Theorem 0.3 (a)
          \item $g \in O(f)$ \hfill 1
          \item $f \in O(g) \land g \in O(f)$ \hfill assumption, 3
          \item $f \in \Theta(g)$ \hfill Theorem 0.3 (b)
          \item $f \in O(g) \cap \Omega(g)$ \hfill Theorem 0.1 (b)
          \item $f \in O(g)$ and $f \in \Omega(g)$ \hfill 6, I2, E9, I2
          \item $f \in \Omega(g) \land f \not\in \Omega(g)$ \hfill assumption, 7
          \item $false$
        \end{enumerate}
        $\therefore O(g) \neq O(f)$ \\
        $\therefore O(f) \subseteq O(g) \land O(g) \neq O(f)$ \\
        $\therefore O(f) \subset O(g)$ \\ 
        \\
        $\therefore O(f) \subset O(g) \Leftrightarrow f \in O(g)$ and $f \not\in \Omega(g) \hfill \qed$
	\section{Question 2}
    Prove that $n! \in o(n^{n})$ by using definition of the $o$-notation \\
    We will prove first that $\forall c \in \mathbb{R}^{+}, c \geq 1, n$!$ < cn^{n}$  $\forall n \geq n_{0}$ \\
    Then we will prove that $\forall c \in \mathbb{R}^{+}, 0 < c < 1, n! < cn^{n}$  $\forall n \geq n_{0}$ \\
    \subsection{$c \geq 1$}
      Proof: (by induction) \\
      \subsubsection{Base Case: n = 2}
        $2! = 2 * 1! = 2 * 1 * 0! = 2 * 1 * 1 = 2 < c2^{2} = c4 \hfill \forall c \geq 1$ \\
        $\therefore$ when $n = 2, n! < cn^{n}$  $\forall c \geq 1$
      \subsubsection{Inductive step}
        Assume $k! < ck^{k}$, $\forall k < n$ \\
        $n$!$ = n * (n - 1)$! \\
        $< n * c * {(n - 1)}^{n - 1}$ \\
        $< c * n * n^{n - 1}$ \\
        $< cn^{n}$ \\
        $\therefore if k! < ck^{k}, \forall k < n$, then $n! < cn^{n}$ \\
        $\therefore$ We have proved the base case and the inductive step. \\
        $\therefore \forall c \in \mathbb{R}^{+}, c \geq 1, n! < cn^{n}$ $\forall n \geq 2$
    \subsection{$0 < c < 1$}
      First we will prove lemma *, which states n! has n + 1 terms involved in its product.
      \subsubsection{lemma *}
        Proof: (by induction) \\
        Base Case: $n = 0$ \\
        $0! = 1 \Rightarrow 0 + 1 terms = n + 1 terms$ \\
        $\therefore$ when $n = 0$, $n$! has n + 1 terms. \\
        Inductive Step: \\
        Assume n! has n + 1 terms \\
        $(n + 1)! = (n + 1) * n$! \\
        $(n + 1) * (n + 1$ terms$)$ \\
        $\therefore (n + 1)$! has $(n + 1) + 1$ terms \\
        $\therefore$ We have proved the base case and the inductive step. \\
        $\therefore \forall n \geq 1, n!$ has $n + 1$ terms $\qed$ \\
        \\
        After proving lemma *, we proceed by noting that since $0 < c < 1$. $\exists m \in \mathbb{R}^{+}, m > 1$ such that $c = \frac{1}{m}$ \\
        Let us choose such a $m$ and then a $n_{0}$ such that $n_{0} = max(3,\lceil m \rceil)$ \\
        We will now prove $n$!$ < cn^{n} \equiv n$!$ < \frac{1}{m}n^{n} \equiv m(n$!$) < n^{n}$ in both $m \leq 3$ and $m > 3$ cases.\\
        \subsubsection{Case 1: $m \leq 3$}
          Proof: (by induction) \\
          Base Case: $n = 3$ \\
          $m \leq 3 \Rightarrow m * 3! = m * 6 < 3 * 3! = 18 < 3^{3} = 27$ \\
          $\therefore$ When $n = 3$, $mn < n^{n}$ \\
          Inductive Step: \\
          Assume $m*n! < n^{n}$ \\
          $m * (n + 1)! = m * (n + 1) * n! = (n + 1) * m * n!$ \\
          $< (n + 1) * n^{n}$ \\
          $< (n + 1) * {(n + 1)}^{n}$ \\
          $= {(n + 1)}^{n + 1}$ \\
          $\therefore m * (n + 1)! < {(n + 1)}^{n + 1}$ \\
          $\therefore$ We have proved the base case and the inductive step. \\
          $\therefore$ When $m \leq 3$, $m(n!) < n^{n}$  $\forall n \geq max(3, \lceil m \rceil)$
          \subsubsection{Case 2: $m > 3$}
            In this case $n \geq m$ \\
            By lemma * above, $n!$ has $n + 1$ terms and $n \geq m > 3$, $n!$ has the form $n!_{n+1 terms} = n * (n - 1) * \dots * 1 * 1$ because $0! = 1! = 1$ \\
            by dropping the two 1's we get $n!_{n-1 terms}$ \\
            Now we note that $m * n!_{n-1 terms}$ has n terms involved, while $n^{n}$ is the product of n multiplied n times, so $n^{n}$ has n terms. \\
            We match the terms up as follows \\
            \begin{table}[h]
              \begin{tabular}{|l|l|l|l|l|l|}
                \hline
                $m * n!_{n-1 terms} =$ & $m$ & $n$  & $(n - 1)$ & $\dots$  & 2 \\ \hline
                $n^{n} =$              & $n$ & $n$  & $n$       & $\dots$  & $n$    \\ \hline
              \end{tabular}
            \end{table} \\
            As can be seen for each term in $m * n!_{n-1 terms}$ we can pair it with a $n$ in $n^{n}$ \\
            $\because n \geq m, \therefore \forall terms \in (m * n!_{n-1 terms})$, each term is less than or equal to its corresponding term in $n^{n}$ \\
            $\because 2 < n, \therefore m * n! < n^{n}$ \\
            $\therefore$ when $m > 3$, $m * n! < n^{n}$  $\forall n \geq max(3,\lceil m \rceil)$ \\
            \\
            $\therefore \forall m > 1, m * n! < n^{n} \equiv n! < cn^{n}$  $\forall n \geq max(3,\lceil m \rceil)$ \\
            $\therefore n! < cn^{n}, \forall c > 0$, and $n \geq max(3,\lceil \frac{1}{c} \rceil) \because 2 < 3$ we can just increase the $n_{0}$ in the proof for $c \geq 1$ \\
            $\therefore \forall c \in \mathbb{R}^{+}, \exists n_{c} \in \mathbb{N}, n! < cn^{n}$, $\forall n \geq n_{c}$ \\
            $\therefore n! \in o(n^{n}) \qed$
  \section{Question 3}
  \begin{center}Prove or disprove that $f(n) + o(f(n)) \in \Theta(f(n))$ \\
  We will prove that \[\lim_{n \to \infty} \frac{f(n) + o(f(n))}{f(n)} = k > 0 \] \end{center}
    
  \begin{center} This will show that $f(n) + o(f(n)) \in \Theta(f(n))$ by Theorem 0.2(c) \\
  First let z(n) = o(f(n)), and note that from Theorem 0.2(d) \end{center}
  \[\lim_{n \to \infty} \frac{z(n)}{f(n)} = 0  \because z(n) \in o(f(n))\]
  Now we show that $f(n) + o(f(n)) \in \Theta(f(n))$ \\
    \[\lim_{n \to \infty} \frac{f(n) + o(f(n))}{f(n)} \\
      = \lim_{n \to \infty} \frac{f(n) + z(n)}{f(n)}  \\
      = \lim_{n \to \infty} \frac{f(n)}{f(n)} + \frac{z(n)}{f(n)} \\
      = \lim_{n \to \infty} \frac{f(n)}{f(n)} + \lim_{n \to \infty} \frac{z(n)}{f(n)} \\
      = (\lim_{n \to \infty} 1) + 0 \\
      = 1
      = k \geq 0
    \]
  \begin{center}$\therefore f(n) + o(f(n)) \in \Theta(f(n)) \qed$\end{center}
    \section{Question 4}
      List the functions below from lowest asymtotic order to highest asymptoic order. If any two (or more) are of the same asymptotic order, indicate which. Provide proofs. \\
      First we will list the order of the functions and then provide a series of proofs that will show that each smaller terms belongs to small-$o$ of the next largest. \\
      This will be true except for the last two which are both constants, we will show in that case that the values of the constants as n goes to infinity gives us the correct order. \\

      \[\prod_{k=2}^{n}(1 - \frac{1}{k^{2}}) < log_{\sqrt{n}} n^{6} < lg^{2} n < 10^{lg lg n} < n^{2} < (n^{2} - 3n)^{3} < n! < n^{n} \]
      \subsection{$n! \in o(n^{n})$}
        In question 2 we showed that $n! \in o(n^{n})$
        $\therefore n! \in o(n^{n})$ $\qed$
        \subsection{($n^{2} - 3n)^{3} \in o(n!)$}
        For all future proofs we will be using Theorem 0.2(d) to prove that each function belongs to $o$ of the next largest. \\
        First let us show the expansion of $(n^{2} - 3n)^{3}$. \\
        $(n^{2} - 3n)^{3} = (n^{2} - 3n)(n^{2} - 3n)^{2} = (n^{2} - 3n)(n^{4} - 6n^{3} + 9n^{2}) = n^{6} - 9n^{5} + 27n^{4} - 27n^{3}$ \\
        Secondly we prove lemma +
        \subsubsection{lemma +}
          Lemma + states that the degree of $n!$ is n $\forall n \geq 1$. \\
          Proof: (by induction) \\
          Base Case: n = 1 \\
          In this case, we must remember that $0! = 1 = (n - 1)!$ \\
          $n! = n * (n - 1)! = n * 0! = n * 1 = n^{1} = 1$ \\
          $\therefore$ when $n = 1$, the degree of n! is 1.
          Assume k! highest degree $\forall k < n$
          Since $(n - 1)!$ has degree $n - 1$, $(n - 1)!$ has the form $(n - 1)! = An^{n - 1} + Bn^{n - 2} + \dots + Z$, where $A,B,\dots,Z \in \mathbb{Z}$ \\
          $n! = n * (n - 1)!$ \\
          $= n * (An^{n -1} + Bn^{n - 2} + \dots + Z)$ \\
          $= An^{n} + Bn^{n - 1} + \dots + Zn$ \\
          $\therefore$, we have proved both the base case, and the inductive step. \\
          $\therefore$, the degree of $n!$ is n. $\forall n \geq 1$  $\qed$. \\          
        \\
        \begin{center}Now we must solve the following\end{center}
          \[\lim_{n \to \infty} \frac{(n^{2} - 3n)^{3}}{n!} = \lim_{n \to \infty} \frac{n^{6} - 9n^{5} + 27n^{4} - 27n^{3}}{n!}\]
          Dividing by the highest power, which as n goes to infinity is n since $n!$ has degree $n$ because of lemma +. \\
          and Since $n!$ is a polynominal of degree n it is has the form $n! = An^{n} + Bn^{n-1} + \dots + Z$, where $A, B, \dots, Z \in \mathbb{Z}$ \\
        Therefore, \[\lim_{n \to \infty} \frac{\frac{n^{6}}{n^{n}} - \frac{9n^{5}}{n^{n}} + \frac{27n^{4}}{n^{n}} - \frac{27n^{3}}{n^{n}}}{\frac{An^{n}}{n^{n}} + \frac{Bn^{n-1}}{n^{n}} + \dots + \frac{Z}{n^{n}}} = \frac{0}{A} = 0\]
        $\therefore (n^{2} - 3n)^{3} \in o(n!)$, by Theorem 2(d). $\qed$
        \subsection{$n^{2} \in o((n^{2} - 3n)^{3})$}
          Again by limits we are to prove that the following limit is equal to zero.
          \[\lim_{n \to \infty} \frac{n^{2}}{(n^{2} - 3n)^{3}} = \lim_{n \to \infty} \frac{n^2}{n^{6} - 9n^{5} + 27n^{4} - 27n^{3}} = \lim_{n \to \infty}\frac{\frac{n^{2}}{n^{6}}}{\frac{n^{6}}{n^{6}} + \frac{9n^{5}}{n^{6}} + \frac{27n^{4}}{n^{6}} + \frac{27n^{3}}{n^{6}}} = \frac{0}{1} = 0\]
          $\therefore n^{2} \in o((n^{2} - 3n)^{6})$, by Theorem 2(d). $\qed$
        \subsection{$10^{lg lg n} \in o(n^{2})$}
        Since $10^{lg lg n} \equiv lg^{lg 10} n$, and by example 2, page 48 in the textbook $lg^{k}n \in o(n^{c})$ $\forall k, c > 0.$ \\
        $\therefore 10^{lg lg n} < o(n^{2})$ $\qed$
        \subsection{$lg^{2}n \in o(10^{lg lg n})$}
        Since $10^{lg lg n} \equiv lg^{lg 10} n = lg^{3 + \epsilon}n,$ for some $\epsilon$. We can show by limits
        \[\lim_{n \to \infty} \frac{lg^{2}n}{lg^{3 + \epsilon}n} = \lim_{n \to \infty} \frac{\frac{lg^{2}n}{lg^{3 + \epsilon}n}}{\frac{lg^{3 + \epsilon}n}{lg^{3 + \epsilon}n}} = \frac{0}{1} = 0\]
        $\therefore lg^{2}n \in o(10^{lg lg n})$, by Theorem 2(d). $\qed$
        \subsection{$log_{\sqrt{n}}n^{6} \in o(lg^{2}n)$}
        Since \[\lim_{n \to \infty}log_{\sqrt{n}}n^{6} = \lim_{n \to \infty}log_{n^{\frac{1}{2}}}n^{\frac{1}{2}^{12}} = \lim_{n \to \infty}12log_{n^{\frac{1}{2}}}n^{\frac{1}{2}} = 12\]
          We can show by limits,
          \[\lim_{n \to \infty}\frac{log_{\sqrt{n}}n^{6}}{lg^{2}n} = \lim_{n \to \infty}\frac{12}{lg^{2}n} = 0\]
          $\therefore log_{\sqrt{n}}n^{6} \in o(lg ^{2}n)$, by Theorem 2(d). $\qed$
          \subsection{$\prod_{k=2}^{n}(1 - \frac{1}{k^{2}}) < log_{\sqrt{n}}n^{6}$}
          First we prove lemma @
          \subsubsection{lemma @}
            We will prove that $\prod_{k=2}^{n}(1 - \frac{1}{k^{2}}) = \frac{n + 1}{2n}$ \\
            Proof: (by induction) \\
            Base case: $n = 2$ \\
            In this case $\prod_{k=2}^{n}(1 - \frac{1}{k^{2}}) = 1 - \frac{1}{2^{2}} = \frac{4 - 1}{4} = \frac{2 + 1}{2*2} = \frac{n + 1}{2n}$ \\
            $\therefore$ When $n = 2, \prod_{k=2}^{n}(1 - \frac{1}{k^{2}}) = \frac{n + 1}{2n}$ \\
            Inductive step: \\
            Assume $\prod_{k=2}^{l}(1 - \frac{1}{k^{2}}) = \frac{l + 1}{2l}, \forall l < n$ \\
            $\prod_{k=2}^{n}(1 - \frac{1}{k^{2}}) = (1 - \frac{1}{n^{2}}) * \prod_{k=2}^{n-1}(1 - \frac{1}{k^{2}}) = (1 - \frac{1}{n^{2}})*\frac{(n - 1) + 1}{2(n - 1)}$ \\
            $= (\frac{n^{2}}{n^{2}} - \frac{1}{n^{2}})*(\frac{n}{2(n-1)}) = (\frac{n^{2} - 1}{n^{2}})*(\frac{n}{2(n - 1)}) = \frac{n^{2} - 1}{2n(n - 1)} = \frac{(n - 1)(n + 1)}{2n(n - 1)}$ \\
            $= \frac{n + 1}{2n}$ \\
            $\therefore$ We have proven the base case, and the inductive step. \\
            $\therefore \prod_{k=2}^{n}(1 - \frac{1}{k^{2}}) = \frac{n + 1}{2n}$, $\forall n \geq 2$ \\
          \\
          Now we prove that $\lim_{n \to \infty}\prod_{k=2}^{n}(1 - \frac{1}{k^{2}}) = \frac{1}{2} < \lim_{n \to \infty}log_{\sqrt{n}}n^{6} = 12$. We show by limits \\
          \[\lim_{n \to \infty}\prod_{k=2}^{n}(1 - \frac{1}{k^{2}}) = \lim_{n \to \infty}\frac{n + 1}{2n} = \lim_{n \to \infty}\frac{\frac{n}{n} + \frac{1}{n}}{\frac{2n}{n}} = \frac{1}{2}\]
      \[\lim_{n \to \infty}log_{\sqrt{n}}n^{6} = \lim_{n \to \infty}12 = 12\]
      Since $\frac{1}{2} < 12$, $\therefore \prod_{k=2}^{n}(1 - \frac{1}{k^{2}}) < \log_{\sqrt{n}}n^{6}$, as n goes to infinity. $\qed$
    \section{Question 5}
      \subsection{Part a}
        We are to solve the recurrence $T(n) = 4T(\frac{n}{2}) + n^{2} + nlgn$ \\
        Note that $a = 4, b = 2, f(n) = n^{2} + nlgn$ \\
        $log_{b}a = log_{2}4 = 2$ \\
        $\therefore f(n) = n^{2} + nlgn = \Theta(n^{2}) = \Theta(n^{log_{b}a}) = \Theta(n^{log_{b}a}lg^{0} n)$ \\
        By Case 2, \\
        We have $T(n) = \Theta(n^{2}lgn) \qed$
      \subsection{Part b}
        We are to solve the recurrence $T(n) = 6T(\frac{n}{2}) + n^{2}lgn$ \\
        Note that $a = 6, b = 6, f(n) = n^{2}lgn$ \\
        $log_{b}a = log_{6}6 = 1$ \\
        $\therefore f(n) = n^{2}lgn = \Omega(n^{2}) = \Omega(n^{1 + 1}) = \Omega(n^{log_{b}a + \epsilon})$ for some $\epsilon > 0$ \\
        Which is Case 3, \\
        however we must prove regularity. So for sufficiently large n \\
        $af(\frac{n}{b}) = 6f(\frac{n}{6})$ \\
        $= 6{(\frac{n}{6})}^{2}lg(\frac{n}{6})$ \\
        $= 6(\frac{n^{2}}{6^{2}})(lgn - lg6)$ \\
        $= \frac{1}{6}n^{2}lgn - \frac{1}{6}n^{2}lg6$ \\
        $\leq cn^{2}lgn$  where $0 < c = \frac{1}{6} < 1$ \\
        $\therefore$ by Case 3, we have $T(n) = \Theta(n^{2}lgn) \qed$
      \subsection{Part c}
        We are to solve the recurrence $T(n) = 3T(\frac{n}{2}) + 2n$ \\
        Note that $a = 3, b = 2, f(n) = 2n$ \\
        $log_{b}a = log_{2}3 > 1$ \\
        $\therefore f(n) = 2n = O(n^{1}) = O(n^{log_{2}3 - \epsilon}) = O(n^{log_{b}a - \epsilon})$, for some $\epsilon \geq log_{2}3 - 1 > 1$\\
        By Case 1, \\
        We have $T(n) = \Theta(n^{log_{2}{3}}) \qed$
    \section{Question 6}
      We are to solve the recurrence $T(n) = T(\frac{n}{2}) + T(\frac{2}{5}n) + n$ \\
      We guess that $T(n) = O(n lg n)$ ie $T(n) \leq cnlgn$ for some constant $c > 0$ and sufficiently large $n$. \\
      \subsection{Inductive Step}
        Assume $T(k) \leq ck lg k$, $\forall k < n$ \\
        Since $\frac{n}{2} < n$, when $n \geq 2$, we have, \\
        $T(\frac{n}{2}) \leq c\frac{n}{2}lg \frac{n}{2}$ \\
        Likewise $\frac{2}{5}n < n$, when $n \geq 3$, we therefore have, \\
        \\
        $T(\frac{2}{5}n) \leq c(\frac{2}{5}n)lg\frac{2}{5}n \leq c\frac{n}{2}lg\frac{n}{2}$,  $\because \frac{2}{5}n \leq \frac{n}{2}$, $\forall n \in \mathbb{N}$ \\
        \\
        $\therefore T(\frac{n}{2}) + T(\frac{2}{5}n) \leq 2c\frac{n}{2}lg \frac{n}{2}$ \\
        Now, \\
        $T(n) = T(\frac{n}{2}) + T(\frac{2}{5}n) + n$ \\
        $\leq 2c\frac{n}{2}lg \frac{n}{2} + n$ \\
        $= cn(lg n - lg 2) + n$ \\
        $= cnlg n - cnlg2 + n$ \\
        $\leq cnlgn$   whenever $c > \frac{1}{lg2}$ \\
      \subsection{Base Case}
        First we note that $n = 1$, gives us $T(1) = a$, where a is a constant. \\
        We note that $T(1) = a \not\leq c(1lg1)$, $\forall a > 0$ \\
        However since we need only prove for sufficiently large $n$ \\
        We note that when we interpret $\frac{n}{2}$ and $\frac{2}{5}n$ as $\lceil\frac{n}{2}\rceil$ and $\lceil\frac{2}{5}n\rceil$ respectively, \\
        that only T(2) directly depends on T(1). We therefore use T(2) as our base case.
        When $n = 2$ \\
        $T(2) = T(\lceil\frac{2}{2}\rceil) + T(\lceil\frac{2*2}{5}\rceil) + 2$ \\
        $= T(1) + T(1) + 2$ \\
        $= 2T(1) + 2$ \\
        $= 2a + 2$ \\
        $2a + 2 \leq c(2lg2)$, whenever $a + 1 \leq c$ \\
        $\therefore$ We have proved the base case, and the inductive step. \\
        $\therefore T(n) \leq cnlgn$  for any $c \geq a + 1$ and for all $n \geq 2$\\
        $\therefore T(n) \in O(nlgn) \qed$
	\end{document}

